\documentclass{article}
\usepackage{fullpage}
\usepackage[MeX]{polski}
\usepackage[utf8]{inputenc}
\usepackage{amsfonts}
\usepackage{amssymb}
\usepackage{graphicx}
\usepackage[pdftex]{color, graphicx}

\author{Piotr Dobrowolski, 291528}
\title{Serwis cytatów -- sentencji \\ Opis protokołu}
\date{\today}
\frenchspacing

\begin{document}
\maketitle
\tableofcontents
\clearpage

\section{Streszczenie}
Niniejszy dokument opisuje specyfikację protokołu serwisu cytatów -- sentencji.
Składa się z następujących sekcji:
\begin{itemize}
\item opis celów protokołu,
\item opis założeń,
\item opis formatu komunikatów,
\item opis stanów.
\end{itemize}


\section{Opis celów protokołu}
Protokół ma obsługiwać sposób szybkiego przekazywania krótkiego tekstu (cytatu/sentencji).
Cytaty będą udostępniane klientom przez serwer.
Protokołu jest nastawiony na szybkość kosztem niezawodności dostarczanych tekstów.


\section{Terminologia}
\begin{description}
\item[Klient: ] użytkownik instancji protokołu, który chce pobrać cytat z bazy utrzymywanej na serwerze,
\item[Serwer: ] aplikacja użytkownika udostępniająca teksty klientom,
\item[Cytat: ] będę używał wymiennie z tekst, sentencja. Są to dane tekstowe przesyłane protokołem,
\item[Długi cytat: ] jest to cytat, który nie mieści się w jednym komunikacie i należy go podzielić.
%może dodać - klient oczekujący, klient rozpoznany, coś takiego?
\end{description}


\section{Założenia}
Serwer odpowiada na żądania klientów. W jednej chwili może obsługiwać wielu na raz.
Oraz może wysyłać wiele różnych cytatów do tego samego klienta w tej samej chwili.

\subsection{Warstwa komunikacji}
Protokół działa w warstwie aplikacji.
\subsection{Wykorzystanie innych protokołów}
Protokół ma działać jak najszybciej, nie gwarantując niezawodności. Używany jest protokół UDP.
\subsection{Model komunikacji}
Protokół działa w modelu klient-serwer. Pojedynczy serwer z cytatami, udostępnia je wielu klientom.
Jak przedstawiono na diagramie poniżej.
Klient nasłuchuje na dynamicznie przydzielonym porcie. Serwer na porcie \verb+s_port+.
\begin{center}
\includegraphics[scale=0.7]{model.png}
\end{center}

\section{Opis formatu komunikatów}
\subsection{Założenia}
\begin{itemize}
\item Protokół stosuje sieciowy porządek oktetów w liczbach,
\item Dopuszczalne uzupełnianie formatu zerami. 
%\item coś jeszcze
\end{itemize}
\subsection{Typy pomocnicze}
\begin{verbatim}
PART_ID_T {
   uint8    part_nr     = 0;
   uint8    all_parts   = 1;
}
\end{verbatim}
Typ ten służy do odróżnienia komunikatu z tekstem, tak aby później można było poskładać długi cytat w dobrej kolejności.\\
Części \verb+part_nr+ numerowane są od $0$ - oznacza numer kolejny części. \verb+all_parts+ - oznacza ilość, na którą został
podzielony cytat.

\begin{verbatim}
QUOT_T {
   PART_ID_T            part_id;
   uint16               quot_length = max_quot_length;
   octet[quot_length]   quot;
}
\end{verbatim}
\verb+quot_length+ - pole zawierające długość cytatu w oktetach.\\
Struktura zachowuje cytat (jedną z części cytatu długiego). Oraz identyfikator jednoznacznie określający która jest to część.

\subsection{Typy komunikatów}
\begin{verbatim}
QUOT_REQUEST_MSG_T {
    uint8   version = version_one;
}
\end{verbatim}
Zgłoszenie się klienta.
Służy do wysłania do serwera prośby o cytat.
Pole \verb+version+ jest to aktualna wersja protokołu aplikacji klienta.
Jest pozostawiona dla przyszłych wersji protokołu.

\begin{verbatim}
QUOT_RESPONSE_MSG_T {
   uint8       id;
   QUOT_T      quotation;
}
\end{verbatim}
Struktura komunikatu wysyłanego z serwera do klienta, \verb+id+ oznacza identyfikator wysłanego
cytatu (aby nie dopuścić do wymieszania części różnych cytatów). Wartość \verb+id+ pozostawiam do wyboru implementującemu
protokół, należy zadbać o różne identyfikatory dla pary klient-serwerem. Pomiędzy klientami
identyfikatory mogą się powtarzać.



\section{Opis wymienianych komunikatów}
Protokół posługuje się dwoma komunikatami: \verb+QUOT_REQUEST_MSG+ i \verb+QUOT_RESPONSE_MSG+. 
(W nawiasach podane są argumenty, a po dwukropku typ komunikatu).
\subsection{Wysyłane przez klienta}
\subsubsection{Zgłaszanie się do serwera}
\verb+QUOT_REQUEST_MSG():     QUOT_REQUEST_MSG_T+\\
Klient posługuje się tym komunikat aby zgłosić chęć pobrania cytatu z serwera.
Po odebraniu komunikatu serwer przygotowuje (losuje, dzieli) cytat i przechodzi do stanu wysyłania cytatu do tego klienta.
\subsection{Wysyłane przez serwer}
\subsubsection{Komunikat z danymi}
\verb+QUOT_RESPONSE_MSG(id, quotation):     QUOT_RESPONSE_MSG_T+\\
W komunikacie zawarty jest:
\begin{itemize}
\item \verb+id:+ identyfikator cytatu, potrzebny w celu jednoznacznej identyfikacji cytatu.
\item \verb+quotation:+ dane, \verb+quotataion.quot:+ zawiera tekst cytatu.
Znaki kodowane są za pomocą systemu UTF-8.
\end{itemize}


\section{Opis stanów}
W protokole występują stany tak jak na schematach poniżej. 
Z założenia pozostawione zostało implementującemu w jaki sposób ma zostać
obsłużona sytuacja braku komunikatu (nie dostarczenia całego 
komunikatu) np. trzykrotne ponowienie pobrania komunikatu. 
Stany protokołu opisują jedną operację pobrania / wysłania komunikatu.
Poniżej rozpatruję stany protokołu występujące po stronie klienta i serwera.
\subsection{Klient}
Klient początkowo zaczyna ze stanu \verb+IDDLE+.
Użytkownik wysyła żądanie do serwera (\verb+QUOT_REQUEST_MSG()+), w chwili oczekiwania jest w
stanie \verb+WAIT_FOR_QUOT+ a następnie zbiera pakiety odebrane z serwera \verb+COLLECT_QUOT+.
Tak działa każde osobne pobranie cytatu.

Klient identyfikuje cytat po odebraniu pierwszej części (zapisuje id). I na podstawie tego
identyfikatora będzie dalej łączył pobrane później części.
Podczas wysłania żądania zostaje zainicjowane odliczanie czasu, jeżeli czas oczekiwania na części cytatu które nie doszły przekroczy 
\verb+timeout+ sekund to pobieranie cytatu zostaje anulowane. \verb+timeout+ ustalane jest przez autora implementacji.
\subsection{Serwer}
Serwer odbiera od klientów żądania \verb+QUOT_REQUEST_MSG()+. Przydziela temu zapytaniu identyfikator.
Następnie dzieli cytat (jeżeli jest długi) zgodnie z \verb+max_quot_length+
i za pomocą komunikatów \\ \verb+QUOT_RESPONSE_MSG(id, quotation)+ wysyła wiadomość do danego klienta.
Działanie serwera może być zakończone wyłącznie za pomocą przerwania.
Wtedy jeżeli serwer jeżeli jest w stanie 'Oczekiwanie na klienta' to nie inicjuje odbierania
\verb+QUOT_REQUEST_MSG+ tylko zakańcza swoje działanie. W każdym innym stanie serwer dokańcza to co rozpoczął --
tak aby znaleźć się w 'Oczekiwanie na klienta'.

Poniższe diagramy przedstawiają opisane wyżej stany.
\begin{center}
\includegraphics[scale=2]{klient.png}
\end{center}
\begin{center}
\includegraphics[scale=2]{serwer.png}
\end{center}


\section{Numery}
\subsection{Stałe}
\begin{itemize}
\item \verb+max_quot_length = 1024;+ ustalany w czasie kompilacji,
\item \verb+timeout = 5;+ proponowany,
\item \verb+s_port = 8078;+
\item \verb+version_one = 1;+
\end{itemize}
\subsection{Kody błędów}
\begin{itemize}
\item \verb+ETIMEDOUT   = 0xffff;+ przekroczenie czasu oczekiwania przez klienta,
\item \verb+ENETUNREACH = 0xfffe;+ sieć jest niedostępna,
\item \verb+EBADMSG     = 0xfffd;+ gdy otrzymany komunikat jest niepoprawny,
\item \verb+ESRVERR     = 0xfffc;+ błąd serwera --- w przypadku, gdy serwer 
        nie może odesłać cytatu (np. gdy serwer nie może pobrać cytatu z bazy)
\end{itemize}

\end{document}
